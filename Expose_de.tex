\documentclass[a4paper,oneside,11pt]{article}
\usepackage{graphicx}
\usepackage{titling}
\usepackage{lipsum}
\usepackage[margin=1.5in,includefoot]{geometry}
\usepackage{amsmath}
\usepackage{fancyhdr}
\pagestyle{fancy}


\title{Titel Expose}
\author{
    Vorname Nachname\\
    {\small Matrikelnummer: 123456}\\
    [1cm]{\small Betreuer: B.Treuer}
}

\date{\today}


\begin{document}
\begin{titlingpage} %This starts the title page
\begin{center}
\includegraphics[height=4cm]{images/Uni_hagen_logo}\\ %Put the logo you want here
\begin{large}
FernUniversit\"at in Hagen \\ %The name your university
Fakult\"at f\"ur Mathematik und Informatik\\
\end{large}
\vspace{4cm} %You can control the vertical distance
\begin{large} 
\textbf{\thetitle} \\
\end{large}
\theauthor\\
\vspace{7cm} %Put the distance you need.
\thedate
\end{center}
\end{titlingpage}

\section{Einf\"uhrung}


\section{Problemstellung}

\subsection{Teilproblem Alpha}

\subsection{Teilproblem Beta}


\section{Zielsetzung}
Im Grunde genommen ist das Ziel nicht der Weg, sondern der Punkt, den wir belegen wollen.

\subsection{Prim\"arziel}
Blindtext \"uber dies und das und so weiter. Unter der Bedingung, dass man sich vorstelle die oben gelegene Hypothenuse hat als Winkel nun 80 Grad, muss man beachten, dass man sich nicht mit zu lange Schachtelsaetzen den Redefluss versaut.
Also macht man lieber kurze Saetze. Die geben dem Leser die Gelegenheit, zu verstehen.

\subsection{Sekund\"arziel}
Wenn man jedoch annimmt, dass blau die bestimmt Farbe des unerlaubten ist, so synergetisiert man immanent, dass die konstante Wahrnehmung von roten Ampeln am oberen Licht bleiben.\\

\begin{center}
$\forall x \in X, \quad \exists y \leq \epsilon$
\end{center}

\section{L\"osungsansatz}


\section{Eigene Motivation}


\section{Ablauf und Meilensteine}


\section{Voraussichtliche Gliederung}

\begin{enumerate}
\item Einleitung
	\begin{enumerate}
	\item Ein Unterpunkt
		\begin{enumerate}
		\item Ein Unter-Unterpunkt
		\end{enumerate}
	\end{enumerate}

\item Grundlagen
\item Ergebnisse
\item Zusammenfassung und Ausblick
\item Anhang

\end{enumerate}

\section*{Literatur}

\end{document}